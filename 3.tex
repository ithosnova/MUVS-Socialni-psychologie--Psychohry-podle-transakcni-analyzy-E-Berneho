\chapter{Transakční analýza}

Transakční analýza oproti ostatním psychologickým pohledům na
člověka se dívá na osobnost člověka z pohledu komunikace, 
z pohledu interpersonálních vztahů mezi jedinci. 
Tyto vztahy jsou opakující se, víceméně trvalé, přímé a 
nezprostředkované mezi dvěma lidmi či malou skupinou lidí a 
současně jsou provázeny emocionálním prožitkem ať už pozitivním či negativním \cite{interpersonalni_vztahy_slovnik}. Základním kamenem transakční analýzy je transakce, což je jednotka komunikace, která se skládá z podnětu a reakce \cite{transakcni_analyza_cata}. Transakce nemusí být verbální, může se jednat i o neverbální komunikaci.

Na počátku teorie transakční analýzy stojí tři základní potřeby:

    \begin{itemize}
        \item hlad po podnětech
        \item hlad po struktuře/organizaci času
        \item hlad po pozici
    \end{itemize}
    
{\it Hlad po podnětech} vede k vyhledávání různých vztahů a sociálních kontaktů. 
Naopak neuspokojení této potřeby může vést k rozvoji duševních poruch až k degenerativním změnám nervových buněk \cite{jak_si_lide_hraji}. 
Uspokojení tohoto hladu vede k interakcím tzv. {\it transakcím}.\\

{\it Hlad po struktuře/organizaci času} způsobuje, že lze veškeré lidské chování a prožívání rozčlenit do sedmi skupin dle struktury času a prostoru při tomto chování. 
Skupiny se liší mírou osobní pohody a společenské prospěšnosti. 
Člověk musí svým interakcím dát nějaký strukturovaný prostor, ve kterém se transakce budou odehrávat. 
Nemá-li čas strukturu, tak se člověk nudí. 
Má-li čas strukturu chaotickou, dostavují se úzkosti a neklid a také negativní transakce.\\

Typy transakcí dle struktury času: \textbf{zůstávání stranou, rituály, tlachání, hry (psychohry), práce, zábava, intimita}. V tomto pořadí jsou seřazeny od nejméně prospěšných po ty nejvíce pohodové \cite{translacni_analyza_prirucka}.\\

\subsubsection*{Zůstávání stranou}
Takový člověk se vyhýbá jakýmkoliv transakcím. Může tak působit povýšeně, ale ve skutečnosti jde o nešťastného člověka, který nedovede najít cestu k druhým.

\subsubsection*{Rituály}
Rituály jsou formalizované zavedené transakce. 
Pravděpodobně lidem šetří energii, současně umožňují se neotevírat se světu, mohou skrývat neupřímnost. 
Jde o formu obrany před intimními transakcemi, které ale působí, že máme o druhé zájem. 
Může se jednat o formu pozdravu nebo o chování při nějaké situaci.

\subsubsection*{Tlachání}
Je sled transakcí, které jsou více spontánní než rituály, ale také zabraňují jedincům ve vzniku intimních transakcí. Forma tlachání se velmi liší v závislosti na pohlaví a věku, ale také na vzdělání. Tyto interakce nemají za cíl vyřešit nějaký problém, ale jen nechat čas volně plynout. Případně ukázat jak máme nad druhým navrh (v čem jsme dobří). 

\subsubsection*{Hry}
Často jsou také označovány jako psychohry, ve většině případů mají negativní, destruktivní charakter. Cílem je získat výhodu nad druhým člověkem. Současně, ale pro některé lidi může být ziskem i pocit zmaru či ponížení. Těmto transakcím se budu dále věnovat v samostatné kapitole.

\subsubsection*{Práce}
Tento typ transakcí bývá pružný, jedinci v něm často zakouší sounáležitost a pozitivní emoce. Cílem transakcí bývá konkrétní vyřešení problémů.

\subsubsection*{Zábava}
Tyto transakce člověku většinou přináší radost, úlohu a odpočinek.
Může se jednat o posezení s přáteli, ale také o práci na zahradě či malování, není nutné, aby u zábavných transakcí byli dva lidé.

\subsubsection*{Intimita}
Tyto typy transakcí jsou pro jedince nejvíce obohacující. 
Jde o interakce s důvěrnými jedinci, o sdílenou vzájemnost, opravdovost a bezprostřednost.\\

\subsection*{Struktura stavů osobnosti}

Základní součástí transakcí a jejich analýzy je dále určení stavu osobnosti. 
Transakční analýza rozlišuje tři ego-stavy: {\it rodič, dospělý a dítě}.




