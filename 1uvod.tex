\chapter{Úvod}

Základem lidského společenství je komunikace. 
Oblast komunikace byla ve dvacátém století předmětem studia mnoha psychologů, sociologů i techniků.
C. E. Shannon a W. Weavera publikovali v roce 1949 matematickou teorii komunikace vyvinutou v rámci laboratoří firmy Bell.  
Na tuto teorii po stránce "sdělení informace" navázal svojí prací G. Gerber a H. D. Lasswell se naopak zabýval efektivitou a účinností komunikace. 
Lasswell se také poprvé pokusil aplikovat Shannonův a Weaverův model na masovou komunikaci.\cite{teorie_komunikace_slovnik}\\

Shannonův a Weaverův model byl lineární. V roce 1953 T. N. Newcomb publikoval svůj nelineární model, který vycházel z představy trojúhelníku. Tento model se snažil ukázat vzájemnou provázanost celého komunikačního systémy. Z těchto prvních modelů se pak teorie komunikace rozdělila na dva směry strukturální teorie a kognitivní a behaviorální teorie.\cite{teorie_komunikace_slovnik}\\

Strukturální teorie se primárně zabývají kódem, jehož prostřednictvím se komunikace odehrává, tedy jazykem. Naproti tomu kognitivní a behaviorální teorie si kladou otázku o významu komunikace. Proč se komunikuje? Mnoho dalších prací se snaží také propojit studium komunikačních procesů se sémantickým studiem významů. \cite{teorie_komunikace_slovnik}\\

Komunikaci resp. interakci mezi jednotlivci popsal v padesátých letech dvacátého století pomocí tzv. transakční analýzy Eric Berne.
Tímto typem studia interakcí se budu zabývat v této práci.
Práce si klade za cíl vysvětlit principy transakční analýzy a souhrnně představit klasifikaci psychoher, které jsou v tomto kontextu chápány jako jeden ze základních pojmů transakční analýzy, jako vzorce chování.

